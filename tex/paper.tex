%% BioMed_Central_Tex_Template_v1.05
%%                                      %
%  bmc_article.tex            ver: 1.05 %
%                                       %


%%%%%%%%%%%%%%%%%%%%%%%%%%%%%%%%%%%%%%%%%
%%                                     %%
%%  LaTeX template for BioMed Central  %%
%%     journal article submissions     %%
%%                                     %%
%%         <27 January 2006>           %%
%%                                     %%
%%                                     %%
%% Uses:                               %%
%% cite.sty, url.sty, bmc_article.cls  %%
%% ifthen.sty. multicol.sty		       %%
%%									   %%
%%                                     %%
%%%%%%%%%%%%%%%%%%%%%%%%%%%%%%%%%%%%%%%%%


%%%%%%%%%%%%%%%%%%%%%%%%%%%%%%%%%%%%%%%%%%%%%%%%%%%%%%%%%%%%%%%%%%%%%
%%                                                                 %%	
%% For instructions on how to fill out this Tex template           %%
%% document please refer to Readme.pdf and the instructions for    %%
%% authors page on the biomed central website                      %%
%% http://www.biomedcentral.com/info/authors/                      %%
%%                                                                 %%
%% Please do not use \input{...} to include other tex files.       %%
%% Submit your LaTeX manuscript as one .tex document.              %%
%%                                                                 %%
%% All additional figures and files should be attached             %%
%% separately and not embedded in the \TeX\ document itself.       %%
%%                                                                 %%
%% BioMed Central currently use the MikTex distribution of         %%
%% TeX for Windows) of TeX and LaTeX.  This is available from      %%
%% http://www.miktex.org                                           %%
%%                                                                 %%
%%%%%%%%%%%%%%%%%%%%%%%%%%%%%%%%%%%%%%%%%%%%%%%%%%%%%%%%%%%%%%%%%%%%%


\NeedsTeXFormat{LaTeX2e}[1995/12/01]
\documentclass[10pt]{bmc_article}    



% Load packages
\usepackage{cite} % Make references as [1-4], not [1,2,3,4]
\usepackage{url}  % Formatting web addresses  
\usepackage{ifthen}  % Conditional 
\usepackage{multicol}   %Columns
\usepackage[utf8]{inputenc} %unicode support
%\usepackage[applemac]{inputenc} %applemac support if unicode package fails
%\usepackage[latin1]{inputenc} %UNIX support if unicode package fails
\urlstyle{rm}
 
 
%%%%%%%%%%%%%%%%%%%%%%%%%%%%%%%%%%%%%%%%%%%%%%%%%	
%%                                             %%
%%  If you wish to display your graphics for   %%
%%  your own use using includegraphic or       %%
%%  includegraphics, then comment out the      %%
%%  following two lines of code.               %%   
%%  NB: These line *must* be included when     %%
%%  submitting to BMC.                         %% 
%%  All figure files must be submitted as      %%
%%  separate graphics through the BMC          %%
%%  submission process, not included in the    %% 
%%  submitted article.                         %% 
%%                                             %%
%%%%%%%%%%%%%%%%%%%%%%%%%%%%%%%%%%%%%%%%%%%%%%%%%                     


%\def\includegraphic{}
%\def\includegraphics{}



\setlength{\topmargin}{0.0cm}
\setlength{\textheight}{21.5cm}
\setlength{\oddsidemargin}{0cm} 
\setlength{\textwidth}{16.5cm}
\setlength{\columnsep}{0.6cm}

\newboolean{publ}

%%%%%%%%%%%%%%%%%%%%%%%%%%%%%%%%%%%%%%%%%%%%%%%%%%
%%                                              %%
%% You may change the following style settings  %%
%% Should you wish to format your article       %%
%% in a publication style for printing out and  %%
%% sharing with colleagues, but ensure that     %%
%% before submitting to BMC that the style is   %%
%% returned to the Review style setting.        %%
%%                                              %%
%%%%%%%%%%%%%%%%%%%%%%%%%%%%%%%%%%%%%%%%%%%%%%%%%%
 

%Review style settings
\newenvironment{bmcformat}{\begin{raggedright}\baselineskip20pt\sloppy\setboolean{publ}{false}}{\end{raggedright}\baselineskip20pt\sloppy}

%Publication style settings
%\newenvironment{bmcformat}{\fussy\setboolean{publ}{true}}{\fussy}



% Begin ...
\begin{document}
\begin{bmcformat}



% This should list the title of the article. The title should include
% the study design, for example:
% 
% A versus B in the treatment of C: a randomized controlled trial
% 
% X is a risk factor for Y: a case control study
%
% The full names, institutional addresses, and e-mail addresses for
% all authors must be included on the title page. The corresponding
% author should also be indicated.


%%%%%%%%%%%%%%%%%%%%%%%%%%%%%%%%%%%%%%%%%%%%%%
%%                                          %%
%% Enter the title of your article here     %%
%%                                          %%
%%%%%%%%%%%%%%%%%%%%%%%%%%%%%%%%%%%%%%%%%%%%%%

\title{An example Bayesian data analysis with PyMC:\\The TFR-HDI controversy}
 
%%%%%%%%%%%%%%%%%%%%%%%%%%%%%%%%%%%%%%%%%%%%%%
%%                                          %%
%% Enter the authors here                   %%
%%                                          %%
%% Ensure \and is entered between all but   %%
%% the last two authors. This will be       %%
%% replaced by a comma in the final article %%
%%                                          %%
%% Ensure there are no trailing spaces at   %% 
%% the ends of the lines                    %%     	
%%                                          %%
%%%%%%%%%%%%%%%%%%%%%%%%%%%%%%%%%%%%%%%%%%%%%%


\author{Abraham D Flaxman\correspondingauthor$^{1}$%
       \email{Abraham D Flaxman\correspondingauthor - abie@uw.edu}%
}
      

%%%%%%%%%%%%%%%%%%%%%%%%%%%%%%%%%%%%%%%%%%%%%%
%%                                          %%
%% Enter the authors' addresses here        %%
%%                                          %%
%%%%%%%%%%%%%%%%%%%%%%%%%%%%%%%%%%%%%%%%%%%%%%

\address{%
    \iid(1)Institute for Health Metrics and Evaluation, Dept of Global Health, Univ of Washington, Seattle, WA USA
}%

\maketitle

%%%%%%%%%%%%%%%%%%%%%%%%%%%%%%%%%%%%%%%%%%%%%%
%%                                          %%
%% The Abstract begins here                 %%
%%                                          %%
%% The Section headings here are those for  %%
%% a Research article submitted to a        %%
%% BMC-Series journal.                      %%  
%%                                          %%
%% If your article is not of this type,     %%
%% then refer to the Instructions for       %%
%% authors on http://www.biomedcentral.com  %%
%% and change the section headings          %%
%% accordingly.                             %%   
%%                                          %%
%%%%%%%%%%%%%%%%%%%%%%%%%%%%%%%%%%%%%%%%%%%%%%
%% The abstract of the manuscript should not exceed 350 words and must
%% be structured into separate sections: Background, the context and
%% purpose of the study; Methods, how the study was performed and
%% statistical tests used; Results, the main findings; Conclusions,
%% brief summary and potential implications. Please minimize the use
%% of abbreviations and do not cite references in the abstract; Trial
%% registration, if your research article reports the results of a
%% controlled health care intervention, please list your trial
%% registry, along with the unique identifying number, e.g. Trial
%% registration: Current Controlled Trials ISRCTN73824458. Please note
%% that there should be no space between the letters and numbers of
%% your trial registration number.

\begin{abstract}
        % Do not use inserted blank lines (ie \\) until main body of text.
        \paragraph*{Background:} Bayesian data analysis is an important tool, but has a steep learning curve. This document provides a rapid introduction through a complete example.
      
        \paragraph*{Methods:} I replicated several analyses of the relationship between Human Development Index (HDI) and Total Fertility Rate (TFR) using the PyMC package for Python.

        \paragraph*{Results:} PyMC provided a simple framework for replicating and extending statistical data analysis in a Bayesian framework.

        \paragraph*{Conclusions:} I hope that this document helps others quickly get started with their own Bayesian data analysis in PyMC.
\end{abstract}



\ifthenelse{\boolean{publ}}{\begin{multicols}{2}}{}




%%%%%%%%%%%%%%%%%%%%%%%%%%%%%%%%%%%%%%%%%%%%%%
%%                                          %%
%% The Main Body begins here                %%
%%                                          %%
%% The Section headings here are those for  %%
%% a Research article submitted to a        %%
%% BMC-Series journal.                      %%  
%%                                          %%
%% If your article is not of this type,     %%
%% then refer to the instructions for       %%
%% authors on:                              %%
%% http://www.biomedcentral.com/info/authors%%
%% and change the section headings          %%
%% accordingly.                             %% 
%%                                          %%
%% See the Results and Discussion section   %%
%% for details on how to create sub-sections%%
%%                                          %%
%% use \cite{...} to cite references        %%
%%  \cite{koon} and                         %%
%%  \cite{oreg,khar,zvai,xjon,schn,pond}    %%
%%  \nocite{smith,marg,hunn,advi,koha,mouse}%%
%%                                          %%
%%%%%%%%%%%%%%%%%%%%%%%%%%%%%%%%%%%%%%%%%%%%%%




%%%%%%%%%%%%%%%%
%% Background %%
%%
%% The background section should be written from the standpoint of
%% researchers without specialist knowledge in that area and must
%% clearly state - and, if helpful, illustrate - the background to the
%% research and its aims. Reports of clinical research should, where
%% appropriate, include a summary of a search of the literature to
%% indicate why this study was necessary and what it aimed to
%% contribute to the field. The section should end with a very brief
%% statement of what is being reported in the article.

\section*{Background}
Bayesian data analysis is an important tool, but it has a steep
learning curve.  MCMC packages like BUGS and PyMC can reduce the
barrier to entry significantly, by removing the burden of coding an
MCMC sampler (e.g. Gibbs sampler) from scratch.  However these systems
still have overhead of their own.

This paper and accompanying source code provide a complete (small)
example of a Bayesian data analysis with PyMC.  In addition to showing
an example of the PyMC syntax necessary, I have also described the
elements of my typical workflow in building, fitting, and validating a
model, as well as some elements of test-driven software development
and debugging PyMC code with the Python debugger.

Wikipedia gives a concise introduction to the topic which this example data analysis will investigate, the so-called \emph{Fertility-development controversy} \ref{FertilityDevelopmentControversy}:
\begin{quote}
The relationship between the total fertility rate (TFR) and
socio-economic development, which is measured by the human development
index (HDI), is the subject of debate in social sciences.

The novelty of this new and on-going debate is to seek a statistically
significant linkage between socio-economic development and total
fertility rate, rather than to examine a more conventional
relationship between fertility and per capita income level, as
discussed in Demographic-economic paradox.

More importantly, the debate has revolved around the fundamental
question in the demographic transition whether there could exist a
so-called ``Fertility J-curve'' in which the fertility declines would
rebound after a certain level (i.e., a threshold) of socio-economic
development.

Conventional wisdom in social sciences has been that developed
countries tend to have lower fertility rates while fertility rates in
developing nations are high. This means that declining fertility rates
are attributed to advances in socio-economic development.

In 2009, a group headed by Mikko Myrskylä of the Max Planck Institute
for Demographic Research proposed that there exists a ``J-shaped''
relationship between human fertility and development — i.e., that
further advances in economic development can reverse the decline in
fertility rate \ref{Myrskyla2009}.

In an article published in Nature, Myrskylä et al. pointed out that
“unprecedented increases” in social and economic development in the
20th century had been accompanied by considerable declines in
population growth rates and fertility. This negative association
between human fertility and socio-economic development has been “one
of the most solidly established and generally accepted empirical
regularities in the social sciences”. The researchers used
cross-sectional and longitudinal analyses to examine the relationship
between total fertility rate (TFR) and the human development index
(HDI).

The main finding of the study was that, in highly developed countries
with HDI above 0.9, further development halts the declining fertility
rates. This means that the previously negative development-fertility
association is reversed; the graph becomes J-shaped. Myrskylä et
al. contend that there has occurred “a fundamental change in the
well-established negative relationship between fertility and
development as the global population entered the twenty-first
century”.

Some researchers doubt J-shaped relationship fertility and
socio-economic development \ref{Luci2010, Furuoka2009}. For example,
Fumitaka Furuoka employed a piecewise regression analysis to
examine the relationship between total fertility rate and human
development index. However, he found no empirical evidence to support
the proposition that advances in development are able to reverse
declining fertility rates.
\end{quote}

My initial investigations of this interesting controversy can be found
on the Healthy Algorithms blog \ref{Flaxman2009}.

%%%%%%%%%%%%%%%%%%
%% Methods
%%
%% This should include the design of the study, the setting, the type
%% of participants or materials involved, a clear description of all
%% interventions and comparisons, and the type of analysis used,
%% including a power calculation if appropriate.

\section*{Methods}
  \subsection*{PyMC project structure}
  My template for a Bayesian data analysis with PyMC separates the
  source code into 4 Python files, which do not correspond precisely
  to the four stages of Bayesian data analysis: data preparation,
  model building, model fitting, model checking.

  The Python files in the src directory of the project template are the following:
\begin{itemize}
\item data.py, where I store all of the data preparation code;
\item models.py, where I store both the model building code and (at
  least initially) the model fitting code
\item graphics.py, where I store all of the code for graphically checking the model fits and producing visual output of model results
\item tests.py, which holds automatic tests for anything at all customized in the code
\end{itemize}

  \subsection*{Data Preparation in data.py}
  File src/data.py holds all of the code for loading and preparing the
  data; it is relatively simple, although it could be simpler.  An
  important task for future labor is building a minimal set of data
  manipulation tools for Python that provides each routine task for
  tabular data manipulaion in a single easy-to-read command.

  For example, the 3 lines necessary to prepare the HDI data are three
  times too many lines, and more than three times too complicated.
  STATA and R provide single-line commands to accomplish this common
  task of translating from a panel data format to an extensive form.

  \subsection*{Yet another sub-heading for this section}
    Text for this sub-section \ldots


 
%%%%%%%%%%%%%%%%%%%%%%%%%%%%
%% Results  %%
%%
%% The Results and Discussion may be combined into a single section or
%% presented separately. Results of statistical analysis should
%% include, where appropriate, relative and absolute risks or risk
%% reductions, and confidence intervals. The results and discussion
%% sections may also be broken into subsections with short,
%% informative headings.

\section*{Results}
  \subsection*{Results sub-heading}
    \subsubsection*{This is a sub-sub-heading}
      Sub-sub-sub-headings are made with the \textsl{\\subsubsection} command. \pb
      pb at end of lines ensures correct paragraph spacing.\pb
	  Text for this sub-sub-section \ldots
    \subsubsection*{Another sub-sub-sub-heading}
      Text for this sub-sub-section \ldots

  \subsection*{Another results sub-heading}
    Text for this sub-section \ldots

  \subsection*{Yet another results sub-heading}
    Text for this sub-section.  More results \ldots

%%%%%%%%%%%%%%%%%%%%%%%%%%%%
%% Discussion
%%
%% The Results and Discussion may be combined into a single section or
%% presented separately. Results of statistical analysis should
%% include, where appropriate, relative and absolute risks or risk
%% reductions, and confidence intervals. The results and discussion
%% sections may also be broken into subsections with short,
%% informative headings.

\section*{Discussion}
  \subsection*{Discussion sub-heading}
    \subsubsection*{This is a sub-sub-heading}
      Sub-sub-sub-headings are made with the \textsl{\\subsubsection} command. \pb
      pb at end of lines ensures correct paragraph spacing.\pb
	  Text for this sub-sub-section \ldots
    \subsubsection*{Another sub-sub-sub-heading}
      Text for this sub-sub-section \ldots

  \subsection*{Another discussion sub-heading}
    Text for this sub-section \ldots

  \subsection*{Yet another discussion sub-heading}
    Text for this sub-section.  More results \ldots


    

%%%%%%%%%%%%%%%%%%%%%%
%% This should state clearly the main conclusions of the research and
%% give a clear explanation of their importance and relevance. Summary
%% illustrations may be included.

\section*{Conclusions}
  Text for this section \ldots


  
%%%%%%%%%%%%%%%%%%%%%%
%% If abbreviations are used in the text, either they should be
%% defined in the text where first used, or a list of abbreviations
%% can be provided, which should precede the competing interests and
%% authors' contributions.

\section*{List of abbreviations used}
  Text for this section \ldots


  
    
%%%%%%%%%%%%%%%%%%%%%%%%%%%%%%%%
%% A competing interest exists when your interpretation of data or
%% presentation of information may be influenced by your personal or
%% financial relationship with other people or organizations. Authors
%% should disclose any financial competing interests but also any
%% non-financial competing interests that may cause them embarrassment
%% were they to become public after the publication of the manuscript.
%%
%% Authors are required to complete a declaration of competing
%% interests. All competing interests that are declared will be listed
%% at the end of published articles. Where an author gives no
%% competing interests, the listing will read 'The author(s) declare
%% that they have no competing interests'.
%%
%% When completing your declaration, please consider the following
%% questions:
%%
%% Financial competing interests
%%
%% * In the past five years have you received reimbursements, fees,
%%   funding, or salary from an organization that may in any way gain or
%%   lose financially from the publication of this manuscript, either
%%   now or in the future? Is such an organization financing this
%%   manuscript (including the article-processing charge)? If so, please
%%   specify.
%% * Do you hold any stocks or shares in an organization that may in
%%   any way gain or lose financially from the publication of this
%%   manuscript, either now or in the future? If so, please specify.
%% * Do you hold or are you currently applying for any patents
%%   relating to the content of the manuscript? Have you received
%%   reimbursements, fees, funding, or salary from an organization
%%   that holds or has applied for patents relating to the content of
%%   the manuscript? If so, please specify.
%% * Do you have any other financial competing interests? If so,
%%   please specify.
%%
%% Non-financial competing interests
%%
%% * Are there any non-financial competing interests (political,
%%   personal, religious, ideological, academic, intellectual,
%%   commercial or any other) to declare in relation to this
%%   manuscript? If so, please specify.
%%
%% If you are unsure as to whether you or one of your co-authors has a
%% competing interest, please discuss it with the editorial office.

\section*{Competing interests }
    Text for this section \ldots

%%%%%%%%%%%%%%%%%%%%%%%%%%%%%%%%
%% 
%% In order to give appropriate credit to each author of a paper, the
%% individual contributions of authors to the manuscript should be
%% specified in this section.

%% An ``author'' is generally considered to be someone who has made
%% substantive intellectual contributions to a published study. To
%% qualify as an author one should 1) have made substantial
%% contributions to conception and design, or acquisition of data, or
%% analysis and interpretation of data; 2) have been involved in
%% drafting the manuscript or revising it critically for important
%% intellectual content; and 3) have given final approval of the
%% version to be published. Each author should have participated
%% sufficiently in the work to take public responsibility for
%% appropriate portions of the content. Acquisition of funding,
%% collection of data, or general supervision of the research group,
%% alone, does not justify authorship.

%% We suggest the following kind of format (please use initials to
%% refer to each author's contribution): AB carried out the molecular
%% genetic studies, participated in the sequence alignment and drafted
%% the manuscript. JY carried out the immunoassays. MT participated in
%% the sequence alignment. ES participated in the design of the study
%% and performed the statistical analysis. FG conceived of the study,
%% and participated in its design and coordination and helped to draft
%% the manuscript. All authors read and approved the final manuscript.

%% All contributors who do not meet the criteria for authorship should
%% be listed in an acknowledgements section. Examples of those who
%% might be acknowledged include a person who provided purely
%% technical help, writing assistance, or a department chair who
%% provided only general support.

\section*{Authors contributions}
    Text for this section \ldots

    
%%%%%%%%%%%%%%%%%%%%%%%%%%%%%%%%
%% You may choose to use this section to include any relevant
%% information about the author(s) that may aid the reader’s
%% interpretation of the article, and understand the standpoint of the
%% author(s). This may include details about the authors'
%% qualifications, current positions they hold at institutions or
%% societies, or any other relevant background information. Please
%% refer to authors using their initials. Note this section should not
%% be used to describe any competing interests.

\section*{Authors information}
    Text for this section \ldots

    

%%%%%%%%%%%%%%%%%%%%%%%%%%%
%% Please acknowledge anyone who contributed towards the study by
%% making substantial contributions to conception, design, acquisition
%% of data, or analysis and interpretation of data, or who was
%% involved in drafting the manuscript or revising it critically for
%% important intellectual content, but who does not meet the criteria
%% for authorship. Please also include their source(s) of
%% funding. Please also acknowledge anyone who contributed materials
%% essential for the study.

%% The role of a medical writer must be included in the
%% acknowledgements section, including their source(s) of funding.

%% Authors should obtain permission to acknowledge from all those
%% mentioned in the Acknowledgements.

%% Please list the source(s) of funding for the study, for each
%% author, and for the manuscript preparation in the acknowledgements
%% section. Authors must describe the role of the funding body, if
%% any, in study design; in the collection, analysis, and
%% interpretation of data; in the writing of the manuscript; and in
%% the decision to submit the manuscript for publication.

\section*{Acknowledgements and Funding}
  \ifthenelse{\boolean{publ}}{\small}{}
  Text for this section \ldots


 
%%%%%%%%%%%%%%%%%%%%%%%%%%%%%%%%%%%%%%%%%%%%%%%%%%%%%%%%%%%%%
%%                  The Bibliography                       %%
%%                                                         %%              
%%  Bmc_article.bst  will be used to                       %%
%%  create a .BBL file for submission, which includes      %%
%%  XML structured for BMC.                                %%
%%                                                         %%
%%                                                         %%
%%  Note that the displayed Bibliography will not          %% 
%%  necessarily be rendered by Latex exactly as specified  %%
%%  in the online Instructions for Authors.                %% 
%%                                                         %%
%%%%%%%%%%%%%%%%%%%%%%%%%%%%%%%%%%%%%%%%%%%%%%%%%%%%%%%%%%%%%


{\ifthenelse{\boolean{publ}}{\footnotesize}{\small}
 \bibliographystyle{bmc_article}  % Style BST file
  \bibliography{bibliography} }     % Bibliography file (usually '*.bib' ) 

%%%%%%%%%%%

\ifthenelse{\boolean{publ}}{\end{multicols}}{}

%%%%%%%%%%%%%%%%%%%%%%%%%%%%%%%%%%%
%%                               %%
%% Figures                       %%
%%                               %%
%% NB: this is for captions and  %%
%% Titles. All graphics must be  %%
%% submitted separately and NOT  %%
%% included in the Tex document  %%
%%                               %%
%%%%%%%%%%%%%%%%%%%%%%%%%%%%%%%%%%%

%%
%% Do not use \listoffigures as most will included as separate files

\section*{Figures}
  \subsection*{Figure 1 - Sample figure title}
      A short description of the figure content
      should go here.

  \subsection*{Figure 2 - Sample figure title}
      Figure legend text.



%%%%%%%%%%%%%%%%%%%%%%%%%%%%%%%%%%%
%%                               %%
%% Tables                        %%
%%                               %%
%%%%%%%%%%%%%%%%%%%%%%%%%%%%%%%%%%%

%% Use of \listoftables is discouraged.
%%
\section*{Tables}
  \subsection*{Table 1 - Sample table title}
    Here is an example of a \emph{small} table in \LaTeX\ using  
    \verb|\tabular{...}|. This is where the description of the table 
    should go. \par \mbox{}
    \par
    \mbox{
      \begin{tabular}{|c|c|c|}
        \hline \multicolumn{3}{|c|}{My Table}\\ \hline
        A1 & B2  & C3 \\ \hline
        A2 & ... & .. \\ \hline
        A3 & ..  & .  \\ \hline
      \end{tabular}
      }
  \subsection*{Table 2 - Sample table title}
    Large tables are attached as separate files but should
    still be described here.



%%%%%%%%%%%%%%%%%%%%%%%%%%%%%%%%%%%
%%                               %%
%% Additional Files              %%
%%                               %%
%%%%%%%%%%%%%%%%%%%%%%%%%%%%%%%%%%%

\section*{Additional Files}
  \subsection*{Additional file 1 --- Sample additional file title}
    Additional file descriptions text (including details of how to
    view the file, if it is in a non-standard format or the file extension).  This might
    refer to a multi-page table or a figure.

  \subsection*{Additional file 2 --- Sample additional file title}
    Additional file descriptions text.


\end{bmcformat}
\end{document}







